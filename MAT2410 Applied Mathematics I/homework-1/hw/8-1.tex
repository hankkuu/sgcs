% Problem 3
\subsubsection{3.} $\mathrm{E}\left(X_1\right)=\mu$, $\mathrm{Var}\left(X_1\right)=4$,
$\mathrm{E}\left(X_2\right)=\mu$, $\mathrm{Var}\left(X_2\right)=7$,
$\mathrm{E}\left(X_3\right)=\mu$, $\mathrm{Var}\left(X_3\right)=14$일 때, 다음 추정량을 이용하여 모평균 $\mu$를 점추정하고자 한다.

\[
    \hat{\mu_1} = \frac{1}{3}\left(X_1+X_2+X_3\right),\qquad
    \hat{\mu_2} = \frac{1}{4}\left(X_1+2X_2+X_3\right),\qquad
    \hat{\mu_3} = \frac{1}{3}\left(2X_1+X_2+2X_3\right)   
\]

\begin{itemize}
    \item[(1)] 각 추정량의 편의를 구하라.
    \item[(2)] 불편추정량과 편의추정량을 구하라.
    \item[(3)] 불편추정량들의 분산을 구하고, 최소분산불편추정량을 구하라.
    \item[(4)] $\mu=1$일 때, 각 추정량들의 평균제곱오차를 구하라.   
\end{itemize}

\paragraph{Solution.}
\begin{itemize}
    \item[(1)] {
        \begin{align*}
            \mathrm{Bias}_{\mu}\left(\hat{\mu_1}\right) &= \frac{1}{3}\left(\mu + \mu + \mu\right) = 0 \\
            \mathrm{Bias}_{\mu}\left(\hat{\mu_2}\right) &= \frac{1}{4}\left(\mu + 2\mu + \mu\right) = 0 \\
            \mathrm{Bias}_{\mu}\left(\hat{\mu_3}\right) &= \frac{1}{3}\left(2\mu + \mu + 2\mu\right) = \frac{2}{3}\mu \\
        \end{align*}
    }
    \item[(2)] {
        (1)에서 불편추정량 $\hat{\mu_1}$, $\hat{\mu_2}$를 얻고, 편의추정량 $\hat{\mu_3}$을 얻는다.
    }
    \item[(3)] {
        \begin{align*}
            \mathrm{Var}\left(\hat{\mu_1}\right) &= \left(\frac{1}{3}\right)^2 \mathrm{Var}\left(X_1+X_2+X_3\right) = 2.778 \\
            \mathrm{Var}\left(\hat{\mu_2}\right) &= \left(\frac{1}{4}\right)^2 \mathrm{Var}\left(X_1+2X_2+X_3\right) = 2.875 \\
            \mathrm{Var}\left(\hat{\mu_3}\right) &= \left(\frac{1}{3}\right)^2 \mathrm{Var}\left(2X_1+X_2+2X_3\right) = 8.778 \\
        \end{align*}
        이고 최소불편추정량 $\hat{\mu_1}$를 얻는다.
    }
    \item[(4)] {
        \begin{align*}
            \mathrm{MSE}\left(\hat{\mu_1}\right) &= \mathrm{Var}\left(\mu_1\right)+\left[\mathrm{Bias}_{\mu}\left(\hat{\mu_1}\right)\right]^2=2.778\\
            \mathrm{MSE}\left(\hat{\mu_2}\right) &= \mathrm{Var}\left(\mu_2\right)+\left[\mathrm{Bias}_{\mu}\left(\hat{\mu_2}\right)\right]^2=2.875\\
            \mathrm{MSE}\left(\hat{\mu_3}\right) &= \mathrm{Var}\left(\mu_3\right)+\left[\mathrm{Bias}_{\mu}\left(\hat{\mu_3}\right)\right]^2=9.222\\
        \end{align*}
    }
\end{itemize}

% Problem 5
\subsubsection{5.} $\mathrm{E}\left(X_1\right)=\mu$, $\mathrm{Var}\left(X_1\right)=2$, 
$\mathrm{E}\left(X_2\right)=\mu$, $\mathrm{Var}\left(X_2\right)=4$이다.

\begin{itemize}
    \item[(1)] $\hat{\mu}=\dfrac{1}{2}\left(X_1+X_2\right)$의 분산을 구하라.
    \item[(2)] $\hat{\mu_1}=aX_1+\left(1-a\right)X_2$의 분산이 최소가 되는 상수 $a$와 최소 분산을 구하라.
\end{itemize}

\paragraph{Solution.}
\begin{itemize}
    \item[(1)] {
        $\mathrm{Var}\left(\hat{\mu}\right) = \left(\dfrac{1}{2}\right)^2\mathrm{Var}\left(X_1+X_2\right) = \dfrac{3}{2}$
    }
    \item[(2)] {
        \begin{align*}
            \mathrm{Var}\left(\hat{\mu_1}\right) &= a^2\mathrm{Var}\left(X_1\right)+\left(a-1\right)^2\mathrm{Var}\left(X_2\right)\\
            &= 2a^2+4\left(a-1\right)^2 \\
            &= \dfrac{2}{3}\left(3a-2\right)^2+\dfrac{4}{3}
        \end{align*}
        이므로 분산이 최소가 될 때 $a=\dfrac{2}{3}$이고, 그 때의 분산은 $\dfrac{4}{3}$이다.
    }
\end{itemize}

% Problem 6
\subsubsection{6.} 크기 20인 표본을 조사한 결과 $\displaystyle \sum_{i=1}^{20}x_i = 48.6$과 
$\displaystyle \sum_{i=1}^{20}x_i^2 = 167.4$인 결과를 얻었다. 이 결과를 이용하여 모평균과 모분산을 점추정하라.

\paragraph{Solution.} 모평균의 점추정값은 $\dfrac{48.6}{20}=2.43$이다.

표본분산은 $\displaystyle \frac{1}{19}\sum_{i=1}^{20} \left(x_i-\overline{x}\right)^2$인데 이는
$\displaystyle \frac{1}{19}\left(\sum_{i=1}^{20} x_i^2-20\overline{x}^2\right)$로도 계산할 수 있다. 따라서 $s^2=2.595$를 얻고,
이 때 표본분산은 모분산에 대한 불편추정량이므로 모분산의 점추정값도 2.595이다.

% Problem 7
\subsubsection{7.} 타이어를 생산하는 어느 회사에서 새로운 제조법에 의하여 생산한 타이어의 주행거리를 알아보기 위하여 7개를 임의로
선정하여 사용한 결과 다음과 같은 결과를 얻었다. 단, 단위는 1,000km이다.

\begin{center}
    \begin{tabular}{ccccccc}
        \hline
        59.2&60.6&56.2&62.0&58.1&57.7&58.1\\
        \hline
    \end{tabular}
\end{center}

\begin{itemize}
    \item[(1)] 이 회사에서 생산된 타이어의 평균 주행거리의 추정값을 구하라.
    \item[(2)] 주행거리의 분산에 대한 추정값을 구하라. 
\end{itemize}

\paragraph{Solution.} 이 표본의 평균 $\overline{x}=58.84$, 표준편차 $s=1.94$이다.
\begin{itemize}
    \item[(1)] {
        표본평균은 모평균에 대한 불편추정량이므로 점추정값은 58.84이다.
    }
    \item[(2)] {
        표본분산은 모분산에 대한 불편추정량이므로 점추정값은 $1.94^2=3.76$이다.
    }
\end{itemize}

% Problem 9
\subsubsection{9.} $X_1$, $X_2$, $\cdots$, $X_{10}$은 $n=10$이고 미지의 $p$를 모수로 갖는 이항분포로부터 추출된 확률표본이라 한다.

\begin{center}
    \begin{tabular}{cccccccccc}
        \hline
        1&8&2&5&7&6&2&9&4&7\\
        \hline
    \end{tabular}
\end{center}

\begin{itemize}
    \item[(1)] $\hat{p}=\dfrac{\overline{X}}{10}$가 모수 $p$에 대한 불편추정량임을 보여라.
    \item[(2)] 관찰된 표본이 위와 같을 때, 모수 $p$에 대한 불편추정값을 구하라. 
\end{itemize}

\paragraph{Solution.}
\begin{itemize}
    \item[(1)] {
        $X_i \sim \mathrm{B}\left(10,\,p\right)$이다. 따라서 $\mathrm{E}\left(X_i\right)=10p$이고, $\mathrm{E}\left(\overline{X}\right)=10p$이다.
        한편 $\mathrm{E}\left(\hat{p}\right)=\mathrm{E}\left(\dfrac{\overline{X}}{10}\right)=p$이므로, $\mathrm{Bias}_p\left(\hat{p}\right)=0$이다.
        따라서 $\hat{p}$는 $p$에 대한 불편추정량이다.
    }
    \item[(2)] {
        표본평균 $\overline{x}=5.1$이므로 $p$의 불편추정값은 0.51이다.
    }
\end{itemize}