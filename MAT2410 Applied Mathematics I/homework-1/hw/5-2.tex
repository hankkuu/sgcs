% Problem 3
\subsubsection{3.} 약속장소에서 친구를 만나기로 하고, 정시에 도착하였으나 친구가 아직 나오지 않았다. 그리고 친구를 만나기 위하여 기다리는 시간은 $\lambda=0.2$인 지수분포에 따른다고 한다.

\begin{itemize}
	\item [(1)] 친구를 만나기 위한 평균 시간을 구하라.
	\item [(2)] 3분이 경과하기 이전에 친구를 만날 확률을 구하라.
	\item [(3)] 10분 이상 기다려야 할 확률을 구하라.
	\item [(4)] 6분이 경과했다고 할 때, 추가적으로 더 기다려야 할 시간에 대한 확률분포를 구하고, 모두 합쳐서 10분 이상 걸릴 확률을 구하라.
\end{itemize}

\paragraph{Solution.} 기다리는 시간을 $X$라고 하자. $X\sim \mathrm{Exp}\left(0.2\right)$이다. 따라서 분포함수 $F\left(x\right)=1-e^{-0.2x}$이다.
\begin{itemize}
	\item [(1)] {
		$\mu = \dfrac{1}{0.2} = 5$
	}
	\item [(2)] {
		\begin{align*}
			P\left(X \leq 3\right) &= F\left(3\right) \\
			&= 1 - e^{-0.2\cdot 3} \approx 0.4512
		\end{align*}
	}
	\item [(3)] {
		\begin{align*}
			P\left(X \geq 10\right) &= 1 - F\left(10\right) \\
			&= e^{-0.2\cdot 10} \approx 0.1353
		\end{align*}
	}
	\item [(4)] {
		지수분포는 무기억성을 가지므로, 추가적으로 기다려야 할 시간은 여전히 $\mathrm{Exp}\left(0.2\right)$를 따른다.
		따라서 모두 합쳐 10분 이상 기다려야 할 확률은 4분 이상 더 기다리는 확률과 같으므로,
		\begin{align*}
			P\left(X \geq 4\right) &= 1 - F\left(4\right) \\
			&= e^{-0.2\cdot 4} \approx 0.4493
		\end{align*}
	}
\end{itemize}

% Problem 5
\subsubsection{5.} 어떤 기계의 고장나는 날의 간격이 $\lambda=0.3$인 지수분포에 따른다고 한다.

\begin{itemize}
	\item [(1)] 이 기계가 고장나는 날의 평균 간격을 구하라.
	\item [(2)] 고장나는 날의 간격에 대한 표준편차를 구하라.
	\item [(3)] 고장나는 날의 간격에 대한 중앙값을 구하라.
	\item [(4)] 이 기계가 수리된 후 다시 고장나기까지 적어도 일주일 이상 사용할 수 있을 확률을 구하라.
	\item [(5)] 이 기계를 5일 동안 정상적으로 사용했을 때, 고장나기까지 적어도 이틀 이상 사용할 수 있을 확률을 구하라.
\end{itemize}

\paragraph{Solution.} 고장나는 날의 간격을 $X \sim \mathrm{Exp}\left(0.3\right)$라고 하자.
\begin{itemize}
	\item [(1)] {
		\[\mu = \frac{1}{0.3} = \frac{10}{3}\]
	}
	\item [(2)] {
		\begin{align*}
			\sigma^2 &= \left(\frac{10}{3}\right)^2 \\
			\Rightarrow \sigma &= \frac{10}{3}
		\end{align*}
	}
	\item [(3)] {
		$X$의 분포함수는 $F\left(x\right)=1-e^{-0.3x}$이다. 중앙값은 $F^{-1}\left(0.5\right)$이므로, $F\left(m\right)=0.5$가 되는 $m$을 구하면
		\begin{align*}
			& F\left(m\right) = 1-e^{-0.3m} = 0.5 \\
			\Rightarrow& e^{-0.3m} = 0.5 \\
			\Rightarrow& -0.3m = -\ln 2 \\
			\Rightarrow& m = \frac{\ln 2}{0.3} \approx 2.310
		\end{align*}
		이다.
	}
	\item [(4)] {
		\begin{align*}
			P\left(X>7\right) &= 1 - F\left(7\right) \\
			&= 1 - \left(1-e^{-0.3\cdot 7}\right) \\
			&= e^{-2.1} \approx 0.1225
		\end{align*}
	}
	\item [(5)] {
		지수분포는 무기억성을 가지므로, 구하는 확률은
		\begin{align*}
			P\left(X>2\right) &= 1 - F\left(2\right) \\
			&= 1 - \left(1-e^{-0.3\cdot 2}\right) \\
			&= e^{-0.6} \approx 0.5488
		\end{align*}
		이다.
	}
\end{itemize}

% Problem 8
\subsubsection{8.} 어느 상점에 매 시간당 평균 30명의 손님이 푸아송 과정을 따라서 찾아온다고 한다.

\begin{itemize}
	\item [(1)] 상점 주인이 처음 두 손님을 맞이하기 위하여 5분 이상 기다릴 확률을 구하라.
	\item [(2)] 처음 두 손님을 맞이하기 위하여 3분에서 5분 정도 기다릴 확률을 구하라.
\end{itemize}

\paragraph{Solution.} 처음 두 손님을 맞이하기까지 걸리는 시간을 $X$라 하면, $X \sim \mathrm{\Gamma}\left(2,\,2\right)$이다.
이 때 확률밀도함수 $f\left(x\right) = \dfrac{x}{4} e^{-\frac{x}{2}}$이다.

\begin{itemize}
	\item [(1)] {
		\begin{align*}
			P\left(X \geq 5\right) &= \int_5^\infty \frac{x}{4} e^{-\frac{x}{2}} \mathop{dx} \\
			&= -\frac{1}{2} \left[\left(x+2\right)e^{-\frac{x}{2}}\right]_5^\infty \\
			&= \frac{7}{2} e^{-\frac{5}{2}} \approx 0.2873
		\end{align*}
	}
	\item [(2)] {
		\begin{align*}
			P\left(3 < X < 5\right) &= \int_3^5 \frac{x}{4} e^{-\frac{x}{2}} \mathop{dx} \\
			&= -\frac{1}{2} \left[\left(x+2\right)e^{-\frac{x}{2}}\right]_3^5 \\
			&= \frac{5}{2} e^{-\frac{3}{2}} - \frac{7}{2} e^{-\frac{5}{2}} \approx 0.2705
		\end{align*}
	}
\end{itemize}

% Problem 9
\subsubsection{9.} 전화 교환대에 1분당 평균 2번의 비율로 신호가 들어오고 있으며, 교환대에 도착한 신호는 푸아송과정을 따른다고 한다.

\begin{itemize}
	\item [(1)] 푸아송 과정의 비율 $\lambda$를 구하라.
	\item [(2)] 교환대에 들어오는 두 신호 사이의 평균 시간을 구하라.
	\item [(3)] 2분과 3분 사이에 신호가 없을 확률을 구하라.
	\item [(4)] 교환원이 교환대에 앉아서 3분 이상 기다려야 첫 번째 신호가 들어올 확률을 구하라.
	\item [(5)] 처음 2분 동안 신호가 없으나 2분과 4분 사이에 4건의 신호가 있을 확률을 구하라.
	\item [(6)] 처음 신호가 15초 이내 들어오고, 그 이후 두 번째 신호가 들어오기까지 3분 이상 걸릴 확률을 구하라.
\end{itemize}

\paragraph{Solution.} 1분간 들어오는 신호의 수를 $X$라고 하자. $X\sim \mathrm{Poi}\left(\lambda\right)$이다.

\begin{itemize}
	\item [(1)] 1분당 평균 2번의 비율로 신호가 들어오므로, $\lambda = 2$이다.
	\item [(2)] 두 신호 사이의 간격 $T$는 $\mathrm{Exp}\left(2\right)$를 따르므로, 이 때 $\mu = 0.5$이다.
	또한 분포함수 $\left(x\right) = 1 - e^{-2x}$이다.
	\item [(3)] {
		2분과 3분 사이에 신호가 없을 확률은 1분간 신호가 없었을 확률과 같으므로,
		\[P\left(T > 1\right) = 1 - F\left(1\right) = e^{-2} \approx 0.1353\]이다.
	}
	\item [(4)] {
		\[P\left(T > 3\right) = 1 - F\left(3\right) = e^{-6} \approx 0.002479\]
	}
	\item [(5)] {
		$Y \sim \mathrm{Poi}\left(4\right)$에 대해 처음 2분 동안 신호가 없을 확률은 $P\left(Y=0\right)$, 2분과 4분 사이에 4건의 신호가 있을 확률은 
		$P\left(Y=4\right)$이므로 동시에 일어날 확률은
		\begin{align*}
			P\left(Y=0\right)P\left(Y=4\right) &= \frac{4^0 e^{-4}}{0!} \times \frac{4^4 e^{-4}}{4!} \\
			&= \frac{4^4 e^{-8}}{4!} \approx 0.003528
		\end{align*}
		이다.
	}
	\item [(6)] {
		\begin{align*}
			P\left(T<0.25\right) P\left(T>3\right) &= \left(1-e^{-0.5}\right)\left[1-\left(1-e^{-6}\right)\right] \\
			&= e^{-6}\left(1-e^{-0.5}\right) \\
			&\approx 0.0009753
		\end{align*}
	}
\end{itemize}

% Problem 16
\subsubsection{16.} 가격이 200(천원)인 프린터의 수명은 평균 2년인 지수분포를 이룬다고 한다. 구매한 날로 1년 안에 프린터가 고장이 나면
제조업자는 구매자에게 전액을 환불하고, 2년 안에 고장이 나면 반액을 환불할 것을 약속하였다. 만일 제조업자가 100대를 판매하였다면, 환불로 인하여 지불해야 할 평균 금액을 구하라.

\paragraph{Solution.} 프린터의 수명을 $X \sim \mathrm{Exp}\left(\dfrac{1}{2}\right)$라고 하면, 누적분포함수 $F\left(x\right) = 1-e^{-\frac{1}{2}x}$이다.
이 때 프린터가 1년 이내에 고장날 확률 $P\left(X\leq1\right) = 1-e^{-\frac{1}{2}}$이고,
1년과 2년 사이에 고장날 확률 $P\left(1<X\leq2\right) = P\left(X\leq2\right) - P\left(X\leq1\right) = e^{-\frac{1}{2}} - e^{-1}$이므로
프린터 한 대에 대한 평균 환불 금액은
\begin{align*}
	& 200 P\left(X\leq1\right) + 100 P\left(1<X\leq2\right) \\
	=& 200 \left(1-e^{-\frac{1}{2}}\right) + 100 \left(e^{-\frac{1}{2}} - e^{-1}\right) \\
	=& 200 - 100 e^{-\frac{1}{2}} - 100 e^{-1} \approx 102.6 \mbox{ (천원)}
\end{align*}
이고, 100대에 대해서는
\[100\left(200 - 100 e^{-\frac{1}{2}} - 100 e^{-1}\right) \approx 10256 \mbox{ (천원)} \]
이 된다. 따라서 평균적으로 10,255,899원을 환불해야 한다.

% Problem 19
\subsubsection{19.} 임의의 양수 $a$, $b$에 대하여 모수 $\lambda$인 지수분포를 갖는 확률변수 $X$는 다음 성질을 가짐을 보여라.
\[P\left(X>a+b\middle|X>a\right) = P\left(X>b\right)\]

\paragraph{Solution.} 지수분포의 누적분포함수 $F\left(x\right) = 1-e^{-\lambda x}$이다. 따라서
\begin{align*}
	P\left(X>a+b\middle|X>a\right) &= \frac{P\left(X>a+b\right)}{P\left(X>a+b\middle|X>a\right)} \\
	&= \frac{1-F\left(a+b\right)}{1-F\left(a\right)} \\
	&= \frac{1-\left(1-e^{-\lambda \left(a+b\right)}\right)}{1-\left(1-e^{-\lambda a}\right)} \\
	&= \frac{e^{-\lambda \left(a+b\right)}}{e^{-\lambda a}} \\
	&= e^{-b} = P\left(X>b\right)
\end{align*}
이다. $\qed$

% Problem 20
\subsubsection{19.} 임의의 양수 $\alpha$에 대하여 $\mathrm{\Gamma}\left(\alpha+1\right)=\alpha\mathrm{\Gamma}\left(\alpha\right)$이
성립한다. 이것을 이용하여, $X\sim\mathrm{\Gamma}\left(\alpha,\,\beta\right)$에 대한 평균과 분산은 각각 $\mu=\alpha\beta$, $\sigma^2=\alpha\beta^2$임을 보여라.

\paragraph{Solution.} $X\sim\mathrm{\Gamma}\left(\alpha,\,\beta\right)$의 확률밀도함수
$f\left(x\right) = \dfrac{1}{\mathrm{\Gamma}\left(\alpha\right)\beta^\alpha}x^{\alpha-1}e^{-\frac{x}{\beta}}$이다.
\begin{align*}
	\mu &= \int_0^\infty x \cdot \dfrac{1}{\mathrm{\Gamma}\left(\alpha\right)\beta^\alpha}x^{\alpha-1}e^{-\frac{x}{\beta}} \mathop{dx} \\
	&= \dfrac{1}{\mathrm{\Gamma}\left(\alpha\right)\beta^\alpha} \int_0^\infty x^{\alpha}e^{-\frac{x}{\beta}} \mathop{dx} \\
	&= \dfrac{\mathrm{\Gamma}\left(\alpha + 1\right)\beta}{\mathrm{\Gamma}\left(\alpha\right)} \int_0^\infty \frac{1}{\mathrm{\Gamma}\left(\alpha+1\right)\beta^{\alpha+1}} x^{\alpha}e^{-\frac{x}{\beta}} \mathop{dx}
\end{align*}
에서 $\dfrac{1}{\mathrm{\Gamma}\left(\alpha+1\right)\beta^{\alpha+1}} x^{\alpha}e^{-\frac{x}{\beta}}$은
$\mathrm{\Gamma}\left(\alpha + 1,\,\beta\right)$의 확률밀도함수이므로 $\displaystyle \int_0^\infty \dfrac{1}{\mathrm{\Gamma}\left(\alpha+1\right)\beta^{\alpha+1}} x^{\alpha}e^{-\frac{x}{\beta}} \mathop{dx} = 1$이고,
감마함수의 정의에 의해 $\dfrac{\mathrm{\Gamma}\left(\alpha + 1\right)}{\mathrm{\Gamma}\left(\alpha\right)} = \alpha$이므로 $\mu = \alpha\beta$이다.

분산을 구하기 위해 $\mathrm{E}\left(X^2\right)$를 계산하면
\begin{align*}
	\mathrm{E}\left(X^2\right) &= \int_0^\infty x^2 \cdot \dfrac{1}{\mathrm{\Gamma}\left(\alpha\right)\beta^\alpha}x^{\alpha-1}e^{-\frac{x}{\beta}} \mathop{dx} \\
	&= \dfrac{1}{\mathrm{\Gamma}\left(\alpha\right)\beta^\alpha} \int_0^\infty x^{\alpha + 1}e^{-\frac{x}{\beta}} \mathop{dx} \\
	&= \dfrac{\mathrm{\Gamma}\left(\alpha+2\right)\beta^2}{\mathrm{\Gamma}\left(\alpha\right)} \int_0^\infty \frac{1}{\mathrm{\Gamma}\left(\alpha+2\right)\beta^{\alpha+2}} x^{\alpha + 1}e^{-\frac{x}{\beta}} \mathop{dx} \\
\end{align*}
에서, $\dfrac{1}{\mathrm{\Gamma}\left(\alpha+2\right)\beta^{\alpha+2}} x^{\alpha + 1}e^{-\frac{x}{\beta}}$은
$\mathrm{\Gamma}\left(\alpha + 2,\,\beta\right)$의 확률밀도함수이므로 $\displaystyle \int_0^\infty \frac{1}{\mathrm{\Gamma}\left(\alpha+2\right)\beta^{\alpha+2}} x^{\alpha + 1}e^{-\frac{x}{\beta}} \mathop{dx} = 1$이다.
따라서
\begin{align*}
	\mathrm{E}\left(X^2\right) &= \dfrac{\mathrm{\Gamma}\left(\alpha+2\right)\beta^2}{\mathrm{\Gamma}\left(\alpha\right)} \times 1 \\
	&= \dfrac{\left(\alpha+1\right)\mathrm{\Gamma}\left(\alpha+1\right)\beta^2}{\mathrm{\Gamma}\left(\alpha\right)} \\
	&= \dfrac{\alpha\left(\alpha+1\right)\mathrm{\Gamma}\left(\alpha\right)\beta^2}{\mathrm{\Gamma}\left(\alpha\right)} = \alpha\left(\alpha+1\right)\beta^2
\end{align*}
이다. 분산 $\sigma^2 = \mathrm{E}\left(X^2\right) - \left[\mathrm{E}\left(X\right)\right]^2$이므로
\begin{align*}
	\sigma^2 &= \alpha\left(\alpha+1\right)\beta^2 - \left(\alpha\beta\right)^2 \\
	&= \alpha\beta^2
\end{align*}
이다. $\qed$